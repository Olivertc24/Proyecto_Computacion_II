\documentclass[12pt, a4paper]{article}

% --- PAQUETES ESENCIALES ---

% Codificación y fuentes
\usepackage[utf8]{inputenc} % Codificación de entrada UTF-8
\usepackage[T1]{fontenc}    % Codificación de fuentes

% Idioma (Español)
\usepackage[spanish]{babel} % Soporte para español, tildes, etc.
\addto\captionsspanish{\renewcommand{\contentsname}{Tabla de Contenido}} % Renombrar 'Índice'

% Geometría y espaciado
\usepackage{geometry}
\geometry{a4paper, left=2.5cm, right=2.5cm, top=3cm, bottom=3cm} % Márgenes
\usepackage{parskip} % Espacio entre párrafos en lugar de sangría

% Gráficos y colores
\usepackage{graphicx}     % Para incluir imágenes
\usepackage{xcolor}       % Para definir colores

% --- CONFIGURACIÓN PARA CÓDIGO SQL (Listings) ---
% Este paquete es crucial para mostrar código
\usepackage{listings}

% Definir los colores para el resaltado de SQL
\definecolor{codegreen}{rgb}{0,0.6,0}      % Comentarios
\definecolor{codeblue}{rgb}{0,0,0.6}       % Palabras clave (Keywords)
\definecolor{codepurple}{rgb}{0.58,0,0.82}  % Strings (Cadenas)
\definecolor{backcolour}{rgb}{0.98,0.98,0.98} % Color de fondo del bloque de código

% Definir el estilo personalizado para SQL
\lstdefinestyle{sqlstyle}{
    language=SQL,                         % Lenguaje SQL
    backgroundcolor=\color{backcolour},   % Fondo
    commentstyle=\color{codegreen},       % Estilo de comentarios
    keywordstyle=\color{codeblue},        % Estilo de palabras clave
    stringstyle=\color{codepurple},       % Estilo de strings
    basicstyle=\ttfamily\footnotesize,    % Fuente monoespaciada pequeña
    breakatwhitespace=false,              % No romper líneas solo en espacios
    breaklines=true,                      % Romper líneas largas
    captionpos=b,                         % Posición del caption (abajo)
    keepspaces=true,                      % Mantener espacios
    numbers=left,                         % Números de línea a la izquierda
    numbersep=5pt,                        % Separación de números
    numberstyle=\tiny\color{gray},        % Estilo de números de línea
    showspaces=false,                     % No mostrar espacios
    showstringspaces=false,               % No mostrar espacios en strings
    showtabs=false,                       % No mostrar tabs
    tabsize=2                             % Tamaño del tab
}

% Establecer el estilo 'sqlstyle' como predeterminado
\lstset{style=sqlstyle}


% --- METADATOS DEL DOCUMENTO ---
\title{Guía de Estudio: \huge \textbf{Bases de Datos Relacionales y SQL}}
\author{Prof. Oliver} % Eres el autor de tu material
\date{\today}


% %%%%%%%%%%%%%%%%%%%%%%%%%%%%%%%%%%%%%%%%%%%%%%%%%%%%%
% % --- INICIO DEL DOCUMENTO ---
% %%%%%%%%%%%%%%%%%%%%%%%%%%%%%%%%%%%%%%%%%%%%%%%%%%%%%
\begin{document}

\maketitle % Crea la portada

\tableofcontents % Crea la tabla de contenido automáticamente

\newpage

% --- CAPÍTULO 1: CONCEPTOS FUNDAMENTALES ---
\section{Conceptos Fundamentales de Bases de Datos}

\subsection{¿Qué es una Base de Datos?}
Una base de datos (BD) es una colección organizada de información o datos estructurados, que normalmente se almacena electrónicamente en un sistema informático. Está diseñada para ser gestionada, actualizada y consultada de manera eficiente.

\subsection{Sistemas Gestores de Bases de Datos (SGBD)}
Un SGBD (en inglés, \textit{Database Management System} o \textbf{DBMS}) es el software que actúa como intermediario entre el usuario y la base de datos. Permite a los usuarios crear, leer, actualizar y eliminar datos (operaciones \textbf{CRUD}).

Ejemplos populares de SGBD Relacionales (RDBMS):
\begin{itemize}
    \item MySQL
    \item PostgreSQL
    \item Microsoft SQL Server
    \item Oracle Database
    \item SQLite (usado en móviles y aplicaciones pequeñas)
\end{itemize}

\subsection{El Modelo Relacional}
Propuesto por Edgar F. Codd en 1970, el modelo relacional organiza los datos en tablas (o "relaciones"). Este es el modelo que estudiaremos y sobre el que opera SQL.
\begin{itemize}
    \item \textbf{Tabla (Relación):} Una colección de datos sobre una entidad específica (ej. "Estudiantes", "Cursos").
    \item \textbf{Fila (Tupla):} Representa un registro único dentro de una tabla (ej. el estudiante "Juan Pérez").
    \item \textbf{Columna (Atributo):} Define una propiedad específica de la entidad (ej. "Nombre", "Fecha\_Nacimiento").
\end{itemize}

% [attachment_0](attachment) % (Descomenta si quieres añadir un gráfico)


\subsection{Claves (Keys) en el Modelo Relacional}
Las claves son fundamentales para garantizar la integridad y conectar las tablas.

\subsubsection{Clave Primaria (Primary Key)}
Es una columna (o conjunto de columnas) que identifica de forma \textbf{única} cada fila de una tabla.
\begin{itemize}
    \item No puede contener valores NULOS.
    \item No puede tener valores duplicados.
    \item Ejemplo: \texttt{ID\_Estudiante} en la tabla \texttt{Estudiantes}.
\end{itemize}

\subsubsection{Clave Foránea (Foreign Key)}
Es una columna en una tabla que hace referencia a la Clave Primaria de otra tabla. Es el mecanismo que \textbf{crea la relación} entre tablas.
\begin{itemize}
    \item Ejemplo: Si tenemos una tabla \texttt{Inscripciones}, esta tendría una columna \texttt{ID\_Estudiante} (que es Clave Foránea) apuntando a la tabla \texttt{Estudiantes}.
\end{itemize}


% --- CAPÍTULO 2: LENGUAJE SQL ---
\section{El Lenguaje SQL (Structured Query Language)}

\subsection{¿Qué es SQL?}
SQL (pronunciado "ese-cu-ele" o "sequel") es el lenguaje estándar utilizado para comunicarse con las bases de datos relacionales. Permite realizar todas las tareas de gestión de datos.

\subsection{Subconjuntos de SQL}
El lenguaje se divide conceptualmente en varias categorías:

\begin{description}
    \item[DDL (Data Definition Language)]
    Lenguaje de Definición de Datos. Se usa para crear y modificar la estructura de la base de datos.
    \begin{itemize}
        \item \texttt{CREATE}: Crea objetos (ej. \texttt{CREATE TABLE...})
        \item \texttt{ALTER}: Modifica la estructura (ej. \texttt{ALTER TABLE... ADD COLUMN...})
        \item \texttt{DROP}: Elimina objetos (ej. \texttt{DROP TABLE...})
    \end{itemize}
    
    \item[DML (Data Manipulation Language)]
    Lenguaje de Manipulación de Datos. Es el más usado; se utiliza para consultar y modificar los datos.
    \begin{itemize}
        \item \texttt{SELECT}: El comando principal para consultar datos.
        \item \texttt{INSERT}: Añade nuevas filas a una tabla.
        \item \texttt{UPDATE}: Modifica filas existentes.
        \item \texttt{DELETE}: Elimina filas.
    \end{itemize}
    
    \item[DCL (Data Control Language)]
    Lenguaje de Control de Datos. Gestiona los permisos y la seguridad.
    \begin{itemize}
        \item \texttt{GRANT}: Otorga permisos a un usuario.
        \item \texttt{REVOKE}: Revoca permisos.
    \end{itemize}
\end{description}


% --- CAPÍTULO 3: ESTRUCTURA BÁSICA DE DATOS (DDL) ---
\section{Definición de Datos (DDL)}
\subsection{Tipos de Datos Comunes}
Antes de crear tablas, debemos conocer los tipos de datos:
\begin{itemize}
    \item \texttt{INT} o \texttt{INTEGER}: Números enteros.
    \item \texttt{VARCHAR(n)}: Texto de longitud variable (hasta 'n' caracteres).
    \item \texttt{CHAR(n)}: Texto de longitud fija.
    \item \texttt{DECIMAL(p, s)}: Números decimales con precisión 'p' y escala 's'.
    \item \texttt{DATE}: Fechas (ej. '2025-11-08').
    \item \texttt{TIMESTAMP}: Fecha y hora.
    \item \texttt{BOOLEAN}: Verdadero o Falso.
\end{itemize}

\subsection{Creación de Tablas (CREATE TABLE)}
Usamos \texttt{CREATE TABLE} para definir una nueva tabla, sus columnas y sus tipos de datos.

\begin{lstlisting}[caption=Ejemplo de creación de dos tablas relacionadas.]
-- Creación de la tabla de Estudiantes
CREATE TABLE Estudiantes (
    ID_Estudiante INT PRIMARY KEY,
    Nombre VARCHAR(100) NOT NULL,
    Email VARCHAR(100) UNIQUE,
    Fecha_Nacimiento DATE
);

-- Creación de la tabla de Cursos
CREATE TABLE Cursos (
    ID_Curso INT PRIMARY KEY,
    Nombre_Curso VARCHAR(150) NOT NULL,
    Creditos INT
);

-- Creación de una tabla de unión (Inscripciones)
CREATE TABLE Inscripciones (
    ID_Inscripcion INT PRIMARY KEY,
    ID_Estudiante INT,
    ID_Curso INT,
    Semestre VARCHAR(20),
    
    -- Definición de las Claves Foráneas
    FOREIGN KEY (ID_Estudiante) REFERENCES Estudiantes(ID_Estudiante),
    FOREIGN KEY (ID_Curso) REFERENCES Cursos(ID_Curso)
);
\end{lstlisting}

\subsection{Modificación de Tablas (ALTER TABLE)}
% ... (Contenido a desarrollar por Oliver) ...

\subsection{Eliminación de Tablas (DROP TABLE)}
% ... (Contenido a desarrollar por Oliver) ...


% --- CAPÍTULO 4: CONSULTAS BÁSICAS (DML) ---
\section{Consultas Básicas (DML - SELECT)}

\subsection{La Cláusula SELECT / FROM}
% ... (Contenido a desarrollar por Oliver) ...
% Explicar SELECT *, SELECT Columna1, Columna2 FROM Tabla...

\subsection{Filtrado de Datos (WHERE)}
% ... (Contenido a desarrollar por Oliver) ...
% Explicar operadores: =, <>, >, <, >=, <=
% Explicar operadores lógicos: AND, OR, NOT
% Explicar: IN, BETWEEN, LIKE, IS NULL

\subsection{Ordenamiento de Resultados (ORDER BY)}
% ... (Contenido a desarrollar por Oliver) ...
% Explicar ASC (default) y DESC

\subsection{Limitación de Resultados (LIMIT / TOP)}
% ... (Contenido a desarrollar por Oliver) ...
% Explicar que varía según el SGBD (LIMIT en MySQL/PostgreSQL, TOP en SQL Server)


% --- CAPÍTULO 5: MANIPULACIÓN DE DATOS (DML) ---
\section{Manipulación de Datos (INSERT, UPDATE, DELETE)}
\subsection{INSERT INTO}
% ... (Contenido a desarrollar por Oliver) ...

\subsection{UPDATE}
% ... (Contenido a desarrollar por Oliver) ...
% ¡Mencionar la importancia crucial de usar WHERE en un UPDATE!

\subsection{DELETE}
% ... (Contenido a desarrollar por Oliver) ...
% ¡Mencionar la importancia crucial de usar WHERE en un DELETE!


% --- CAPÍTULO 6: TEMAS AVANZADOS ---
\section{Consultas Avanzadas y JOINs}

\subsection{Funciones de Agregación}
% ... (Contenido a desarrollar por Oliver) ...
% Explicar: COUNT(), SUM(), AVG(), MIN(), MAX()

\subsection{Agrupamiento de Datos (GROUP BY)}
% ... (Contenido a desarrollar por Oliver) ...
% Explicar cómo funciona con las funciones de agregación

\subsection{Filtrado de Grupos (HAVING)}
% ... (Contenido a desarrollar por Oliver) ...
% Diferencia clave entre WHERE (filtra filas) y HAVING (filtra grupos)

\subsection{Unión de Tablas (JOINs)}
%  % (Recomendado añadir un diagrama de Venn aquí)
% ... (Contenido a desarrollar por Oliver) ...
\subsubsection{INNER JOIN}
\subsubsection{LEFT (OUTER) JOIN}
\subsubsection{RIGHT (OUTER) JOIN}
\subsubsection{FULL (OUTER) JOIN}

\subsection{Subconsultas (Subqueries)}
% ... (Contenido a desarrollar por Oliver) ...
% Explicar subconsultas en el SELECT, FROM y WHERE.

\end{document}

% %%%%%%%%%%%%%%%%%%%%%%%%%%%%%%%%%%%%%%%%%%%%%%%%%%%%%
% % --- FIN DEL DOCUMENTO ---
% %%%%%%%%%%%%%%%%%%%%%%%%%%%%%%%%%%%%%%%%%%%%%%%%%%%%%
